1.  Metodología usada. 
Para el desarrollo completo del juego se utilizará Agile Development ( Desarrollo Ágil)  en este caso, se utilizará para que en cada meta se pueda establecer del juego a un estado cerca al entregable siempre.

Además, se usará un framework Scrum para trabajar en la ceación del producto final. Con esdte framework se pretende crear un proceso de desarrollo auto manejado, iterativo e incremental; permitiendo así establecer un aumento en la productividad al desarrollar el producto final.

Utilizando Scrum, el juego será desarrollado en iteraciones, también llamadas sprints, basadas en progresos que tardaran cuatro días cada uno; estas iteraciones son reducidas con lo usual del framework (2 a 4 semanas) debido a los tiempos entre entregas y el trabajo necesario para llevar a cabo las metas en cada iteración que permiten ver el progreso en cada una. 

El proceso estará definido por zonas importantes que deben ser reconocidas:
1.  Se establece un Product Backlog, en el que se creará una lista de tareas basadas en prioridad que se deben realizar. Cada una de esas tareas será llamada PBI o Product Backlog Item.
2.-  Cada una de las PBI se debe evaluar en conjunto para establecer las tareas requeridas para implementarla. Esas tareas, pasaran a ser el Sprint Backlog. 
3.-  Se programará una conferencia de 15 minutos donde se mostrarán los progresos realizados y los impedimentos para realizarlos.
4.- Diariamente, esto es al final de cada Sprint,  se realizarán en la noche Nightlys o commits nocturnos del trabajo realizado. En este commit se debe entregar una versión del juego que pueda ser ejecutado y jugado dentro de las tareas establecidas en el Sprint Backlog.
5.- Finalmente, se realiza una reunión de revisión del sprint donde se evaluará el desempeño y el logro del sprint. Luego, se creará el sprint backlog del próximo día.

Logros con la Metodología Scrum

1. Tiempos (TimeBoxing)
Scrum al ser iterativo entrega resultados bajo una regularidad. Esto permite a los desarrolladores y directores a sincronizar el trabajo y guiarlo lo mejor posible para alcanzar la meta en un tiempo establecido.
2. Prioridades
Al establecer prioridades, algunas de las funcionalidades requieren estar preparadas primero. Por lo tanto, en lugar de implementar todo lo necesario en el documento de diseño, se establecen cuales son más importantes y se maneja de forma adecuada las cargas en los equipos de manera sincronizada, es decir sin dejar a uno de los desarrolladores sin trabajo o la espera de que otro termine.
3.- Organización Propia
Al dejar delegadas tareas a cada uno de los desarrolladores; cada uno tiene un amplio margen de como utilizar el sprint para entregar el producto requerido en las mejores condiciones y en el tiempo descrito.

Product Backlog

Entorno de Ventana
Area de Juego
Movimiento
Sistema de Puntaje
HUD de estado del juego
Perks o Modificadores de Juego
Texturas
Entorno de Juego
Modelos del Juego
Otros Entornos
Animaciones

Roles
ScrumMaster: Mateo Florido Sanchez
-Tareas: Mejora el uso del Scrum en el equipo mediante el acompañamiento, monitoría y eliminación rápida de problemas que aparten al equipo de la entrega de valor.
Se asegura de que los imperdimentos sean solucionados e.g. bugs, personas que no realizan progreso de la tarea asignada y retrasa la entrega del Sprint.
Equipo: Mateo Florido Sanchez, Nicolas ?
-Tareas Entregar al final de cada Sprint las tareas designadas. 
Reportar entrega cada día antes de los Nightly's para llevar a cabo las pruebas de CI y Testing

Planeamiento del Sprint

El ScrumMaster ayuda al equipo a identificar restricciones que puedan alterar la entrega del Sprint e.g. problemas con el anterior sprint
El equipo toma un PBI con la prioridad más alta y lo divide en tareas que ayuden a crear el Sprint Backlog
El equipo define el tiempo necesario para completar cada tarea y verifica si las horas necesarias exceden la capacidad del equipo.
Se añaden las tareas al sprint backlog y se verifica si se ha completado.
***AÑADIR LA GRAFICA DE LA PAG 78***

Diariamente se llevará una gráfica de Burndown en la que el equipo definirá cuantas horas faltan para acabar el Sprint definido. Así se puede reconocer el progreso del equipo.

Además de lo anterior, se usará un tablero de tareas en una plataforma online. Se compone de cuatro columnas: en la primera se ven los PBI en orden prioritario definido por el equipo y el ScrumMaster; en la segunda, todas las tareas aún no iniciadas de cada PBI; en la tercera, todas las tareas que estén en progreso y por último, las tareas que ya están completadas.
***AÑADIR GRAFICA***



